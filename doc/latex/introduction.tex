% Copyright (c)  2005  Bram de Greve.
% 
% Permission is granted to copy, distribute and/or modify 
% this document under the terms of the GNU Free Documentation 
% License, Version 1.2 or any later version published by the 
% Free Software Foundation; with the Invariant Sections being 
% the chapter ``Introduction", no Front-Cover Texts, and no 
% Back-Cover Texts.  A copy of the license is included in the 
% section entitled "GNU Free Documentation License".

\chapter{Introduction}

\section{What is LiAR?}

LiAR is a recursive acronym that stands for \emph{LiAR isn't A Raytracer}\footnote{pun intended. Oh, for heaven's sakes! LiAR really \emph{is} a ray tracer.  I just thought it was a pretty neat idea to call it a liar since the goal is to make photorealistic renders of something that doesn't exist after all.  Actually, I had this name before I started coding it.  To be more precise, I started coding this ray tracer because I felt someone had to write a ray tracer with that name.  Whatever.  Enjoy!~:)}.  It is an open source and freeware raytracer that is mostly written in \Cpp  as an extension module for \emph{Python}~\cite{Python}.  It's goals are to be physically correct and to be easily extended.  Well, we'll see about \emph{that}, won't we?~:)

It is mostly based on one article, one book and one specification.  The article is \emph{The ray tracing kernel}~\cite{kirk88ray}, on which the overall design of LiAR is mostly based.  Most apparent is the fact that the core module of LiAR is called \kw{kernel}.  The book is \emph{Physically based rendering}~\cite{Pharr2004pbrt} and has an enormous amount of implementation details. It is by far out the best book on the planet on ray tracing.  So obviously, a lot is inspired by it, especially the sampling mechanism.  As last, the \emph{RenderMan Interface Specification}~\cite{RISpec} helped to get an idea of what was to be implemented to get a full feature set.  

Furthermore, the whole thing is written on top of \emph{Lass}~\cite{Lass}, a \Cpp library which happens to share an author with LiAR.  This is of course an entire coincidence.

\section{The guide, the reference and the manual}

This document is intended for both the user and the developer of LiAR.  It tends to be a reference of what's available.


\section{Software license}

LiAR is release under the GNU General Public License (GPL) what means it's \emph{free}\footnote{free as in freedom, not as in beer} software.  You can use it, copy it, modify it and redistribute it again as many times as you like, as long as you respect the terms of the GPL.  More information about the GPL and the full license text can be found the GNU website~\cite{GPL}.

\begin{footnotesize}
\begin{verbatim}
LiAR isn't a raytracer
Copyright (C) 2004-2005  Bram de Greve

This program is free software; you can redistribute it and/or modify
it under the terms of the GNU General Public License as published by
the Free Software Foundation; either version 2 of the License, or
(at your option) any later version.

This program is distributed in the hope that it will be useful,
but WITHOUT ANY WARRANTY; without even the implied warranty of
MERCHANTABILITY or FITNESS FOR A PARTICULAR PURPOSE.  See the
GNU General Public License for more details.

You should have received a copy of the GNU General Public License
along with this program; if not, write to the Free Software
Foundation, Inc., 59 Temple Place, Suite 330, Boston, MA  02111-1307  USA
\end{verbatim}
\end{footnotesize}